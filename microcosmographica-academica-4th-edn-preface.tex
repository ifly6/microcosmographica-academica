When \ma\ was reprinted in 1922, Cornford himself wrote a short preface on the subject of its continued relevance after fourteen years and a war. (It is to this preface by the way, that we owe the admirable definition of Propaganda as "that branch of the art of lying which consists in very nearly deceiving your friends without quite deceiving your enemies".) It may be supposed that, had he lived, he would have added a few more words in explanation of its reappearance after another twenty-seven years and the upheaval of a second and greater war. If I may attempt to fill a gap which only he could have filled adequately, I would say first that the justification of \mashort\ does not lie only in its relevance to the present situation. It is al\-ready a classic. The academic scene is indeed changed since 1908, but it is only an additional pleasure to come unexpectedly upon such historical details as the prohibition of walking to Madingley on Sundays without academical dress. It will not deprive a new reader of this pleasure by mentioning others. It seems perhaps more serious that whereas Cornford's enemy was inertia ("There is only one argument for ding something; the rest are arguments for doing nothing"), we may reasonable hold today that that the greatest peril to the things which we value (and he valued) lies in too rapid change. Unfortunately too, we can no longer confine to the unregenerate Adullamites the description that they are "dangerous, because they know what they want; and that is, all the money that is going". Yet after all, in so far as  they are a menace, the changes of today are for the most part not caused by academics themselves. They tend to originate outside, and is not the reason for their success very largely the persistence of the old arguments for doing nothing about it among those who, given a few more of the qualities of the Young Man in a Hurry, might have saved the situation?

I started by affirming the appeal of \mashort\ does not depend on its continued relevance to the current situation, but have drifted into maintaining that relevance persists. This I profoundly believe. No one who has served on the College Governing Body or the Faculty Board can read without immediate recognition the chapter on Argument and the Conduct of Business. Cornford mentions the applicability of his principles to Government departments during the first world war. I myself can vouch for the delight with which they were received in the second by those of my colleagues who seemed worthy to be introduced to them. Shortly before the war moreover I tried the book on the head of an electrical engineering firm, and he assured me that the business world itself was in urgent need of its counsel. Nor is it idle to mention that in 1945 the publishers received a request from the University of Chicago Press to print a small private edition to be given away to a select number of friends. The Argument of the Wedge, or the Principle of Unripe Time, cannot become out of date. They have their roots in no changing historical situation, but in human nature. Read and see.

\begin{flushright}
	W K C Guthrie
\end{flushright}

Cambridge, 1949\footnote{This preface was first printed in the fourth edition (1949); it was reprinted in the sixth edition (1964) as well.}